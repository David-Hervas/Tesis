\chapter[Challenges for analyzing metabolomics data. Strategies and aims of the thesis]{Challenges for analyzing metabolomics data. Strategies and aims of the thesis}
\label{chapter:challenges}

\section{Challenges for analyzing metabolomic data}
\label{sec:challengesmetabodata}
Traditionally, 'omic' data analyses have been performed using classical statistical methods from last century. In most cases, these methods consisted in bivariate tests such as \textit{t test} or \textit{Wilcoxon-Mann Whitney test}, followed by some sort of multiple comparissons correction such as \textit{Bonferroni} or \textit{False Discovery Rate} \parencite{hochberg1990more, benjamini1995controlling}. These approaches suffer from several drawbacks including lack of statistical power, lack of interpretability of results, and omission of complex relationships among variables \parencite{strasak2007statistical}. Thus, classical statistical methods based on bivariate tests are not adequate to extract all the information available in 'omic' data sets.
Some of the mentioned problems such as the lack of interpretability or the omission of complex relationships could be adressed using statistical models such as linear models or generalized linear models, but these methods also suffer from other problems when dealing with 'omic' data, such as the large number of variables and the low sample size, which produces overfitting, and the high correlation among variables, which produces multicollinearity. All these limitations associated with 'omic' data analysis have motivated the development of numerous novel statistical techniques during the last decades specifically aimed at solving them. Prediction methods such as PLS \parencite{wold1984collinearity}, lasso \parencite{tibshirani1996regression}, elastic net \parencite{zou2005regularization} and random forest \parencite{breiman2001random}, among others, can deal with most of the problems exposed in the preceding lines. Nevertheless, other challenges remain partially unresolved, such as the analysis of repeated measures data or, more generally, data organized in multiway arrays. Also, in a discipline where the number of variables is so large, the development of variable selection methods is of utmost importance.

\section{Strategies and aims of the thesis}
\label{sec:strategiesthesis}
As presented in the previous sections, 'omic' data sets and, more specifically, metabolomic data sets contain a large amount of information in the form of huge data structures. The magnitude of this data sets dificults their handling as well as their analysis and interpretation. Additionally, metabolomics presents specific singularities that make necessary the use of complex pre-processing and analysis techniques. The aim of metabolomic analyses is the identification of those metabolites related to the specific biological question being studied. As of today, the majority of statistical analyses performed on metabolomic data sets are based on classical methods or, in the best cases, on adequate but limited modern statistical methods. Among these adequate methods stand out principal component analysis (PCA), partial least squares (PLS), penalization methods such as ridge or lasso and other machine learning techniques such as random forest or boosting. However, these techniques are not able no deal effectively with all the particularities of metabolomic data sets. Their limitations get clearly exposed when dealing with data sets with repeated measurements over time or, more generally, three-way or multi-way data sets. An important part of the work on this thesis will consist on critically reviewing an assessing the usefulness of the different presented tools for metabolomic data analysis.

Some useful tool for analyzing three-way data are the Tucker3 and PARAFAC models and, when some \textbf{Y} data structure is to be predicted, the $N$-PLS model. Related to the problem of these datasets with a large number of features, comes the issue of variable selection. Variable selection is essential for facilitating e.g. biological interpretation of the results when analyzing '-omic' data sets. It is often the case that the aim of these analyses is to find a new biomarker or a specific set of biomarkers, also called signature, to diagnose or predict the onset of a disease. For this cases, the $N$-PLS algorithm does not provide variable selection implemented within the algorithm. Although some methods have been developed to perform it after the fitting of the model. One of the main research lines of this thesis will be the introduction of L1-penalization in the $N$-PLS algorithm to allow for variable selection within the model-fitting step. This approach should not only facilitate interpretation by producing a reduced model including fewer variables, but should also reduce prediction error by completely eliminating noise features instead of downweighting them as $N$-PLS does. 
The full list of objectives for this thesis is presented below:

\begin{itemize}
    \item Try, assess and evaluate different preprocessing techniques for metabolomic data
    \item Study and critically evaluate the use of different modern data analysis techniques for dealing with metabolomic data sets.
    \item Develop novel algorithms for embedding L1-penalization with $N$-PLS for the analysis of three- or multi-way datasets that allow for variable selection at the model-fitting step
    \item Develop a software package based in the R language, that implements the new algorithm along with all the extra functions needed for performing complete analyses of three-way data sets and their posterior interpretation.
\end{itemize}