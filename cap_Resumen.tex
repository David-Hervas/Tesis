% ---------------------------------------------------------------------
% ---------------------------------------------------------------------
% ---------------------------------------------------------------------

\chapter*{Resumen}

% ---------------------------------------------------------------------
% ---------------------------------------------------------------------
% ---------------------------------------------------------------------

En las últimas décadas los avances tecnológicos han tenido como consecuencia la generación de una creciente cantidad de datos en el campo de la biología y la biomedicina. A día de hoy, las así llamadas tecnologías “ómicas”, como la genómica, epigenómica, transcriptómica o metabolómica entre otras, producen bases de datos con cientos, miles o incluso millones de variables. 

El análisis de datos ómicos presenta una serie de complejidades tanto metodológicas como computacionales que han llevado a una revolución en el desarrollo de nuevos métodos estadísticos específicamente diseñados para tratar con este tipo de datos. 

A estas complejidades metodológicas hay que añadir que, en la mayor parte de los casos, las restricciones logísticas y/o económicas de los proyectos de investigación suelen conllevar que los tamaños muestrales en estas bases de datos con tantas variables sean muy bajos, lo cual no hace sino empeorar las dificultades de análisis, ya que se tienen muchísimas más variables que observaciones.

Entre las técnicas desarrolladas para tratar con este tipo de datos podemos encontrar algunas basadas en la penalización de los coeficientes, como lasso o elastic net, otras basadas en técnicas de proyección sobre estructuras latentes como PCA o PLS y otras basadas en árboles o combinaciones de árboles como random forest.

Todas estas técnicas funcionan muy bien sobre distritos datos ómicos presentados en forma de matriz (IxJ), sin embargo, en ocasiones los datos ómicos pueden estar expandidos, por ejemplo, al tomar medidas repetidas en el tiempo sobre los mismos individuos, encontrándonos con estructuras de datos que ya no son matrices, sino arrays tridimensionales o three-way (IxJxK). En estos casos, la mayoría de las técnicas citadas pierden toda o buena parte de su aplicabilidad, quedando muy pocas opciones viables para el análisis de este tipo de estructuras de datos.

Una de las técnicas que sí es útil para el análisis de estructuras three-way es $N$-PLS, que permite ajustar modelos predictivos razonablemente precisos, así como interpretarlos mediante distintos gráficos.

Sin embargo, relacionado con el problema de la escasez de tamaño muestral relativa al desorbitado número de variables, aparece la necesidad de realizar una selección de variables relacionadas con nuestra respuesta. Esto es especialmente cierto en el ámbito de la biología y la biomedicina, ya que no solo queremos poder predecir lo que va a suceder, sino entender por qué sucede, qué variables están implicadas y, a poder ser, no tener que volver a recoger los cientos de miles de variables para realizar una nueva predicción, sino utilizar unas cuantas, las más importantes, para poder diseñar kits predictivos coste/efectivos de utilidad real. Por ello, el objetivo principal de esta tesis es mejorar las técnicas existentes para el análisis de datos ómicos, específicamente las encaminadas a analizar datos three-way, incorporando la capacidad de selección de variables, mejorando la capacidad predictiva y mejorando la interpretabilidad de los resultados obtenidos. Todo ello se implementará además en un paquete de R completamente documentado, que incluirá todas las funciones necesarias para llevar a cabo análisis completos de datos three-way. 

El trabajo incluido en esta tesis por tanto, consta de una primera parte teórico-conceptual de desarrollo de la idea del algoritmo, asi como su puesta a punto, validación y comprobación de su eficacia, de una segunda parte empírico-práctica de comparación de los resultados del algoritmo con otras metodologías de selección de variables existentes como VIP y de una parte adicional de programación y desarrollo de software en la que se presenta todo el desarrollo del paquete de R, su funcionalidad y capacidades de análisis. 

El desarrollo y validación de la técnica, así como la publicación del paquete de R, que ya cuenta con varios cientos de usuarios, ha permitido ampliar las opciones actuales para el análisis de datos ómicos three-way  abriendo un gran número de líneas futuras de investigación.


\ifEBOOKPDF
	\bigskip
\else
	\vfill
\fi


\chapter*{Resum}

% ---------------------------------------------------------------------
% ---------------------------------------------------------------------
% ---------------------------------------------------------------------

En les últimes dècades els avançaments tecnològics han tingut com a conseqüència la generació d'una creixent quantitat de dades en el camp de la biologia i la biomedicina. A dia d'avui, les anomenades tecnologies "òmiques", com la genòmica, epigenòmica, transcriptòmica o metabolòmica entre altres, produeixen bases de dades amb centenars, milers o fins i tot milions de variables. 

L'anàlisi de dades òmics presenta una sèrie de complexitats tant metodològiques com computacionals que han portat a una revolució en el desenvolupament de nous mètodes estadístics específicament dissenyats per a tractar amb aquest tipus de dades.

A aquestes complexitats metodològiques cal afegir que, en la major part dels casos, les restriccions logístiques i / o econòmiques dels projectes de recerca solen comportar que les magnituts de les mostres en aquestes bases de dades amb tantes variables siguen molt baixes, el que no fa sinó empitjorar les dificultats d'anàlisi, ja que es tenen moltíssimes més variables que observacions

Totes aquestes tècniques funcionen molt bé sobre diferents dades òmics presentats en forma de matriu (IXJ), però, en ocasions les dades òmics poden estar expandits, per exemple, cuan ni ha mesures repetides en el temps sobre els mateixos individus, trobant-nos amb estructures de dades que ja no són matrius, sinó arrays tridimensionals o three-way (IxJxK). En aquestos casos, la majoria de les tècniques mencionades perden tota o bona part de la seua aplicabilitat, quedant molt poques opcions viables per a l'anàlisi d'aquest tipus d'estructures de dades.

Una de les tècniques que sí que és útil per a l'anàlisi d'estructures three-way és $N$-PLS, que permet ajustar models predictius raonablement precisos, així com interpretar-los mitjançant diferents gràfics.

No obstant això, relacionat amb el problema de l'escassetat de mostres relativa al desorbitat nombre de variables, apareix la necessitat de realitzar una selecció de variables relacionades amb la nostra resposta. Això és especialment cert en l'àmbit de la biologia i la biomedicina, ja que no només volem poder predir el que va a succeir, sinó entendre per què passa, quines variables estan implicades i, si pot ser, no haver de tornar a recollir els centenars de milers de variables per realitzar una nova predicció, sinó utilitzar unes quantes, les més importants, per poder dissenyar kits predictius cost / efectius d'utilitat real. Per això, l'objectiu principal d'aquesta tesi és millorar les tècniques existents per a l'anàlisi de dades òmics, específicament les encaminades a analitzar dades three-way, incorporant la capacitat de selecció de variables, millorant la capacitat predictiva i millorant la interpretabilitat dels resultats obtinguts. Tot això s'implementarà a més en un paquet de R completament documentat, que inclourà totes les funcions necessàries per a dur a terme anàlisis completes de dades three-way.

El treball inclòs en aquesta tesi per tant, consta d'una primera part teorica-conceptual de desenvolupament de la idea de l'algoritme, així com la seua posada a punt, validació i comprovació de la seua eficàcia, d'una segona part empíric-pràctica de comparació dels resultats de l'algoritme amb altres metodologies de selecció de variables existents com VIP i d'una part adicional de programació i desenvolupament de programació en la qual es presenta tot el desenvolupament del paquet de R, la seua funcionalitat i capacitats d'anàlisi.

El desenvolupament i validació de la tècnica, així com la publicació del paquet de R, que ja compta amb diversos centenars d'usuaris, ha permès ampliar les opcions actuals per a l'anàlisi de dades òmics three-way obrint un gran nombre de línies futures de recerca .


\ifEBOOKPDF
	\bigskip
\else
	\vfill
\fi


\chapter*{Abstract}

% ---------------------------------------------------------------------
% ---------------------------------------------------------------------
% ---------------------------------------------------------------------

In the last decades, advances in technology have enabled the gathering of an increasingly amount of data in the field of biology and biomedicine. The so called "-omics" technologies such as genomics, epigenomics, transcriptomics or metabolomics, among others, produce hundreds, thousands or even millions of variables per data set.

The analysis of omic data presents different complexities that can be methodological and computational. This has driven a revolution in the development of new statistical methods specifically designed for dealing with these type of data. 

To this methodological complexities one must add the logistic and economic restrictions usually present in scientific research projects that lead to small sample sizes paired to these wide data sets. This makes the analyses even harder, since there is a problem in having many more variables than observations.

Among the methods developed to deal with these type of data we can find some based on the penalization of the coefficients, such as lasso or elastic net, others based on projection techniques, such as PCA or PLS, and others based in regression or classification trees and ensemble methods such as random forest.

All these techniques work fine when dealing with different omic data in matrix format (\textit{I}x\textit{J}), but sometimes, these \textit{I}x\textit{J} data sets can be expanded by taking, for example, repeated measurements at different time points for each individual, thus having \textit{I}x\textit{J}x\textit{K} data sets that raise more methodological complications to the analyses. These data sets are called three-way data. In this cases, the majority of the cited techniques lose all of a good part of their applicability, leaving very few viable options for the analysis of this type of data structures.

One useful tool for analyzing three-way data, when some \textbf{Y} data structure is to be predicted, is $N$-PLS. $N$-PLS reduces the inclusion of noise in the models and obtains more robust parameters when compared to PLS while, at the same time, producing easy-to-understand plots.

Related to the problem of small sample sizes and exorbitant variable numbers, comes the issue of variable selection. Variable selection is essential for facilitating biological interpretation of the results when analyzing 'omic' data sets. Often, the aim of the study is not only predicting the outcome, but also understanding why it is happening and also what variables are involved. It is also of interest being able to perform new predictions without having to collect all the variables again. Because all of this, the main goal of this thesis is to improve the existing methods for 'omic' data analysis, specifically those for dealing with three-way data, incorporating the ability of variable selection, improving predictive capacity and interpretability of results. All this will be implemented in a fully documented \texttt{R} package, that will include all the necessary functions for performing complete analyses of three-way data.

The work included in this thesis consists in a first theoretical-conceptual part where the idea and development of the algorithm takes place, as well as its tuning, validation and assessment of its performance. Then, a second empirical-practical part comes where the algorithm is compared to other variable selection methodologies such as VIP. Finally, an additional programming and software development part is presented where all the \texttt{R} package development takes place, and its functionality and capabilities are exposed.

The development and validation of the technique, as well as the publication of the R package, which has already hundreds of users, has opened many future research lines.

\ifEBOOKPDF
	\bigskip
\else
	\vfill
\fi
% ---------------------------------------------------------------------

