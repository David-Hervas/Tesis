% ---------------------------------------------------------------------
% ---------------------------------------------------------------------
% ---------------------------------------------------------------------

\chapter{La familia de normas ISO 9000 y los equipos de medida}

% ---------------------------------------------------------------------
% ---------------------------------------------------------------------
% ---------------------------------------------------------------------

\section{La familia de normas ISO 9000}

El término ISO 9000 se utiliza normalmente para referirse a un conjunto completo de cinco documentos numerados desde ISO 9000 hasta ISO 9004, \parencite{UNE-EN-ISO-9000}, y que de forma colectiva exponen procedimientos diseñados para conseguir el \textbf{aseguramiento de la calidad}. Estas normas imponen a los proveedores de bienes y servicios el requisito de establecer y mantener un sistema económico, eficiente y demostrable que asegure que su producto o servicio es conforme a los requisitos especificados para el mismo.

El primer documento, ISO 9000, no es realmente una norma en sí misma, sino que más bien consiste en una serie de directrices para la selección y uso de los documentos ISO 9001, ISO 9002 e ISO 9003. Estos tres documentos son las normas de aseguramiento de la calidad más aplicados actualmente. El conjunto se completa con el documento ISO 9004, que, de nuevo, no se trata de una norma en sí misma, sino un documento que proporciona directrices para el desarrollo e implantación de sistemas de calidad.

Las normas de calidad de ISO han sido adoptadas por la mayoría de países de todo el mundo, pero generalmente se publican en cada país con denominaciones y códigos ligeramente distintos. Por ejemplo, en España son publicadas por AENOR bajo la denominación de \textbf{UNE-EN-ISO 9000}.

\begin{parrafoDestacado}
Es importante resaltar que el conjunto de normas de calidad ISO 9000 define qué elementos debe contener un sistema de la calidad, pero no prescribe cómo se deben implantar estos elementos en ninguna situación particular o caso concreto. Esto es necesariamente así debido a que cada situación y cada aplicación es diferente.
\end{parrafoDestacado}

% ---------------------------------------------------------------------
% ---------------------------------------------------------------------
% ---------------------------------------------------------------------

\section
	[Requisitos para los equipos de medida]
	{Requisitos relativos a los equipos de inspección, medición y ensayo}

De acuerdo con el número de junio de 1992 de \ingles{Quality System Update}, las cinco razones principales que suelen producir problemas en las empresas que desean implantar la ISO 9000 son:

\begin{itemize}

	\item
	Control de la documentación.
	\item 
	Calibración.
	\item
	Seguimiento de los equipos de medida.
	\item
	Registros de formación del personal.
	\item
	Planificación de contactos con los proveedores.

\end{itemize}

Esto muestra hasta qué punto resulta de gran importancia aquellos aspectos relacionados con los equipos de medida, y en particular, con la calibración de los mismos, \parencite{griful1998control}.

Existe un documento complementario, véase \parencite{UNE-EN-30012-1}, codificado como ISO 10012-1 que define con más detalle los procedimientos necesarios para seleccionar, utilizar, calibrar, controlar y mantener equipos de medida, tal como marca ISO 9001-9003.

A continuación se resumen los principales requisitos de calibración\index{Calibración!Requisitos} y medida contenidos en el documento ISO 10012-1:

\begin{enumerate}

%\renewcommand{\labelitemi}{\ding{221}}
%\renewcommand{\labelenumi}{\textbf{\theenumi})}

\item
La compañía debe disponer de \textbf{equipos de medida} para cuantificar todos los parámetros relacionados con la calidad, y éstos equipos deben tener las características metrológicas adecuadas, \parencite{lope1998instrumentos}.

\item
Debe estar documentada la \textbf{lista de todos los instrumentos} utilizados para cuantificar los parámetros relacionados con la calidad.

\item
Se debe implantar y mantener un sistema para el \textbf{control} y la \textbf{calibración} de los equipos de medida.

\item
Todos los equipos utilizados para realizar medidas de la calidad, y todos los equipos utilizados para calibrar, se deben \textbf{manipular con cuidado} y deben ser usados de tal forma que su exactitud y ajuste quede a salvo.

\item
Todas las medidas, tanto para calibrar equipos como para la verificación del producto, deben realizarse teniendo en cuenta todos los \textbf{errores e incertidumbres} significativos identificados en el proceso de medida.

\item
El cliente debe tener acceso a \textbf{pruebas objetivas} de que el sistema de medida es efectivo.

\item
La calibración se debe realizar con equipos con \textbf{trazabilidad\index{Calibración!Trazabilidad} a patrones nacionales}.

\item
Todas las personas que desarrollan funciones de calibración deben estar \textbf{debidamente formadas}.

\item
Los procedimientos de calibración deben estar \textbf{documentados}.

\item
El sistema de calibración debe ser \textbf{revisado} periódica y sistemáticamente para asegurar que continúa siendo efectivo.

\item
Se debe mantener una \textbf{ficha o registro} de calibración para cada equipo de medida por separado. Cada ficha debe demostrar que el instrumento es capaz de realizar medidas dentro de los límites designados. Estas fichas deben contener, al menos, esta información:
	
	\begin{itemize}
		\item 
		Una descripción del instrumento y una identificación única.
		\item
		La fecha de calibración.
		\item
		Los resultados de la calibración.
		\item
		El intervalo de calibración, además de la fecha de la próxima  calibración.
	\end{itemize}

\item
Dependiendo del tipo de instrumento a calibrar, también se debe incluir parte o toda la información que se relaciona a continuación: 

	\begin{itemize}
	
		\item
		El procedimiento\index{Calibración!Procedimiento de} de calibración.
		\item
		Los límites de error permisibles (ver \autoref{sec:equipos_de_medida:caracteristicas}).
		\item
		Informe de todos los efectos acumulativos de incertidumbre en los datos de calibración.
		\item
		Las condiciones medioambientales requeridas para la calibración.
		\item
		La fuente que certifica la trazabilidad empleada.
		\item
		Los detalles de cualquier reparación o modificación que pudiera afectar el estado de la calibración.
		\item
		Cualquier limitación de uso del instrumento.
	
	\end{itemize}

\item
Cada instrumento debe estar \textbf{etiquetado}, de manera que se muestre el estado de calibración y cualquier limitación de uso (únicamente donde es posible).

\item
Cualquier instrumento que haya fallado, que sea sospecho o se sepa que se encuentra fuera de calibración, debe ser \textbf{retirado del uso} y etiquetado visiblemente para prevenir posibles usos accidentales del mismo.

\item
Los equipos de medida ajustables se deben \textbf{sellar} para evitar manipulaciones no deseadas.

\end{enumerate}

% ---------------------------------------------------------------------
% ---------------------------------------------------------------------
% ---------------------------------------------------------------------

\section{Interpretación de los requisitos}

Para cumplir con la norma, de acuerdo con \citeauthor{morris1997measurement} \parencite*{morris1997measurement}, se hace necesario implantar y mantener un sistema de medida y control de la calidad que asegure que la calidad de los bienes fabricados o servicios no se desvían de los límites de error establecidos.

Los límites de error se deben establecer en función de la situación. Cuando un producto o un servicio está específicamente diseñado para un cliente, los niveles de calidad adecuados son los acordados con el cliente. Esto puede ser escrito, en algunos casos, en el acuerdo contractual entre el proveedor y el cliente.

En algunas situaciones, se deben aplicar normas legales que deben ser cumplidas. Por ejemplo, las balanzas de peso para su uso comercial deben cumplir normas de exactitud publicadas.

En otros casos, se aplican normas consensuadas normalmente por asociaciones de organizaciones comerciales. Si no se cumplen ninguno de los casos anteriores, el proveedor debe evaluar cuál es el cliente medio dentro del mercado al cual va dirigido su producto, para establecer los límites de error.

Una vez que el fabricante o proveedor de un servicio ha fijado los límites de error adecuados, debe implantar un sistema que mida el producto a intervalos de tiempo convenientes y que asegure que éste no se sale de los límites de error establecidos.

Al realizar estas medidas, todos los instrumentos utilizados deben estar calibrados a intervalos de tiempo apropiados para asegurar la precisión de la medida realizada, de acuerdo con los procedimientos expuestos en la norma. Todos estos procedimientos para la realización de medidas sobre el producto y calibración de equipos de medidas deben estar completamente documentados, y esta documentación debe ponerse a disposición de los clientes si es requerida.

La fuente más común de dificultad para conseguir la conformidad con la norma es la interpretación de los requisitos para proveer y mantener equipos de calibración y medida. La calibración de los equipos de medida asegura que la exactitud de la medida de cada instrumento involucrado en el proceso de medida es conocida a lo largo de todos su rango de medida, cuando se utiliza bajo determinadas condiciones ambientales.

Esta información se obtiene por comparación de la salida del instrumento a calibrar con la salida de un instrumento de exactitud conocida, al aplicar la misma entrada a ambos. Pero las características metrológicas de un equipo de medida cambian con las condiciones externas, por lo tanto, es necesario cuantificar el efecto de las condiciones ambientales sobre el funcionamiento de este equipo de medida.

Por otra parte, las características metrológicas de un equipo de medida no permanecen constantes en el tiempo, la calibración de los equipos a intervalos de tiempo determinados se hace necesaria. La frecuencia de calibración de un equipo de medida puede variar en función de los resultados que se van obteniendo y de la información de que se dispone sobre el equipo.

% ---------------------------------------------------------------------
% ---------------------------------------------------------------------
% ---------------------------------------------------------------------

\section{La certificación}

Se entiende por certificación\index{Certificación} ``La actividad que permite establecer la conformidad de una determinada empresa producto o servicio con los requisitos definidos en normas o especificaciones técnicas'' (LEY 21/1992, de 16 de julio, de Industria).

Existen dos tipos de certificación: 

\begin{itemize}
	\item
	Voluntaria: productos, sistemas de la calidad, procesos, servicios.
	\item
	Obligatoria: derivada de algún reglamento técnico.
\end{itemize}

Mediante la certificación de sistemas de la calidad, el Organismo de Certificación\index{Organismos de Certificación} declara haber obtenido la confianza adecuada en la conformidad del sistema de la calidad de la empresa, debidamente identificada, con algún modelo de sistema de la calidad.

Los Organismos de Certificación deben desarrollar esta actividad con imparcialidad, transparencia y objetividad, disponiendo para ello de procedimientos para la certificación de productos, servicios y sistemas de la calidad. Estos procedimientos describen los procesos de concesión de la certificación.

Existen numerosas organizaciones que certifican Sistemas\index{Sistema de Aseguramiento de la Calidad} de Aseguramiento de la Calidad. Un número importante de ellas se hallan acreditadas, a través de la Entidad Nacional de Acreditación (ENAC). La acreditación puede entenderse como un reconocimiento formal de la capacidad técnica de certificar. De este modo se garantiza su capacidad técnica frente a posibles clientes, otras organizaciones y la administración. En el catálogo editado por ENAC o en su página web pueden consultarse los Organismos de Certificación actualmente acreditados.

% ---------------------------------------------------------------------
% ---------------------------------------------------------------------
% ---------------------------------------------------------------------

\section{Los equipos de medida para procesos industriales}
\label{sec:equipos_de_medida}


% ---------------------------------------------------------------------
% ---------------------------------------------------------------------
\subsection{Introducción}

Los equipos de medida\index{Equipo de medida} se encargan de realizar mediciones sobre las variables involucradas en los procesos industriales. A partir de ellos, se observa y se controla el proceso. Dichas mediciones deben ser fiables, seguras y de gran exactitud, y en general permitir la visualización continua del proceso.

\begin{parrafoDestacado}
Los requisitos técnicos de un proceso industrial y/o de sus resultados (productos y servicios) en todas las etapas de su ciclo de vida (comercialización, diseño, fabricación, montaje, etc.) se establecen mediante especificaciones que definen intervalos de valores admisibles o tolerancias para las diferentes magnitudes que determinan su calidad.
\end{parrafoDestacado}

Cada vez que hay que decidir si el valor de una característica está dentro de la tolerancia especificada, es preciso medir con suficiente exactitud, fiabilidad y seguridad como para tomar esta decisión con la menor incertidumbre compatible con los condicionantes económicos.


% ---------------------------------------------------------------------
% ---------------------------------------------------------------------
\subsection{Características de los equipos de medida}
\label{sec:equipos_de_medida:caracteristicas}

Cada aplicación de un equipo de medida requiere de una exactitud y de unas prestaciones distintas. Si se pretendiera exigir mayor confianza a la medida que la necesaria, el coste del proceso de medida se vería incrementado sustancialmente. 

La elección, por tanto, se debe realizar a partir del conocimiento de las características, tanto estáticas como dinámicas, que definen el funcionamiento de estos equipos.

\begin{parrafoDestacado}
En las hojas de especificaciones técnicas del fabricante que acompañan al equipo pueden encontrarse las características que presenta el instrumento bajo condiciones normales de calibración.

Es responsabilidad del personal técnico asegurar que la información suministrada por el fabricante sea suficiente para la aplicación.
\end{parrafoDestacado}

% ---------------------------------------------------------------------
\subsubsection{Rango de medida (\ingles{range})}

El rango\index{Rango de medida} define los valores mínimo o límite inferior (\ingles{lower range limit}) y máximo o límite superior (\ingles{upper range limit}) de lectura para los cuales el equipo ha sido diseñado.


% ---------------------------------------------------------------------
\subsubsection{Alcance (\ingles{span})}

El alcance\index{Alcance} es la diferencia entre el valor máximo y el mínimo de la variable de entrada del instrumento de medida. Hay que destacar que muchos equipos presentan un alcance que puede ser ajustado según los requisitos de la señal (\ingles{calibrated span}). En este caso el alcance puede no coincidir con los valores que definen su rango.

% ---------------------------------------------------------------------
\subsubsection{Fondo de escala (\ingles{full-scale reading})}

Máximo valor que puede medir el instrumento o del que se obtiene lectura. 

% ---------------------------------------------------------------------
\subsubsection{Exactitud (\ingles{accuracy})}

Es\index{Exactitud} la capacidad de un equipo de medida de dar indicaciones que se aproximen al verdadero valor de la magnitud medida. Para expresar esto, se indica el intervalo dentro del cual puede recaer el valor real del mensurando. Se debe evitar traducirlo como ``precisión'', ya que el término \ingles{precision} en inglés denota otro significado, como se verá a continuación. La exactitud es un parámetro determinante para la elección de un equipo u otro.

% ---------------------------------------------------------------------
\subsubsection{Tolerancia (\ingles{tolerance})}

La tolerancia\index{Tolerancia} es un término íntimamente relacionado con la exactitud y define el máximo error esperado en cierto valor. Estrictamente hablando, no es una característica estática del instrumento de medida. La tolerancia, cuando se emplea de forma apropiada, hace en realidad referencia a la desviación de un producto fabricado respecto a un valor especificado.

% ---------------------------------------------------------------------
\subsubsection{Fidelidad (\ingles{precision})}

La fidelidad\index{Fidelidad} es la cualidad que caracteriza la capacidad del instrumento de medida para dar el mismo valor de magnitud al medir varias veces en unas mismas condiciones. Un instrumento con fidelidad alta implica que, al tomar muchas medidas, la dispersión en éstas es muy baja. Esta característica no guarda ninguna relación con la exactitud del instrumento.

% ---------------------------------------------------------------------
\subsubsection{Repetibilidad (\ingles{repeatability})}

La repetibilidad\index{Repetibilidad} tiene un significado similar a la fidelidad, si bien se entiende ahora que las medidas son realizadas en un periodo de tiempo corto y, por tanto, en unas condiciones ambientales similares.

% ---------------------------------------------------------------------
\subsubsection{Reproducibilidad (\ingles{reproducibility})}

La Reproducibilidad\index{Reproducibilidad} es un término equivalente a la fidelidad cuando las medidas son tomadas de manera que entre cada una de ellas se producen cambios en las condiciones ambientales, en el observador, en la localización y ubicación o en los intervalos de medida.

% ---------------------------------------------------------------------
\subsubsection{Desplazamiento (\ingles{bias}, \ingles{offset})}

Un desplazamiento en la medida se produce cuando existe un error constante sobre todo el rango de medida. Este error generalmente puede ser eliminado por medio de un procedimiento de ajuste (ajuste de cero).

% ---------------------------------------------------------------------
\subsubsection{Linealidad (\ingles{linearity})}

Generalmente se desea que la lectura de los equipos de medida sea linealmente proporcional a la cantidad medida. Esto significa que debe ser posible trazar una línea recta que haga corresponder cada valor de la cantidad medida con la lectura de salida. 

La no linealidad\index{Linealidad} del equipo queda definida como la máxima desviación (o residuo) de las lecturas respecto a dicha recta.

% ---------------------------------------------------------------------
\subsubsection
	[Sensibilidad de la medida]
	{Sensibilidad de la medida (\ingles{sensitivity of measurement})}

La sensibilidad\index{Sensibilidad} de la medida es la variación relativa de la salida del instrumento frente a un incremento en la cantidad medida.

% ---------------------------------------------------------------------
\subsubsection
	[Sensibilidad ante perturbaciones]
	{Sensibilidad ante perturbaciones (\ingles{sensitivity to disturbance})}

Todas las especificaciones indicadas en la hoja del fabricante, o bien obtenidas por calibración de un equipo de medida, sólo son válidas para condiciones normales controladas de temperatura, presión, etc. Si tienen lugar cambios en esas condiciones, las características estáticas del instrumento pueden sufrir variaciones. Estas alteraciones pueden modificar las características del instrumento de dos formas:

{\samepage\begin{itemize}

\item
Deriva de paso por cero (\ingles{zero drift}/\ingles{offset}): Se trata de una lenta variación con el tiempo del valor de paso por cero. Este cambio generalmente tiene lugar como consecuencia de una variación de temperatura. El efecto que trae asociado es un desplazamiento en la lectura.

\item
Deriva de la sensibilidad (\ingles{sensitivity drift}/\ingles{scale factor drift}): es la variación que tiene lugar en la sensibilidad del instrumento como consecuencia de un cambio en las condiciones ambientales.

\end{itemize}}


% ---------------------------------------------------------------------
\subsubsection{Histéresis (\ingles{hysteresis})}

Por histéresis\index{histéresis} se entiende la propiedad presente en algunos instrumentos de medida que provoca que la curva de medida difiera según las lecturas se hagan de forma ascendente o en sentido descendente. 

Los parámetros que permiten cuantificar esta característica son la histéresis máxima de entrada y la histéresis máxima de salida.

% ---------------------------------------------------------------------
\subsubsection{Zona muerta (\ingles{dead space})}

La zona muerta\index{Zona muerta} de un instrumento se define como el rango de entrada para el cual no se obtiene lectura en la salida. Todo instrumento con histéresis va a presentar (en promedio) también zona muerta. Otros equipos, aún sin tener histéresis, pueden presentar zona muerta.

