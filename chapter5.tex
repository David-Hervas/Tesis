% ---------------------------------------------------------------------
% ---------------------------------------------------------------------
% ---------------------------------------------------------------------

\chapter[Sparse N-PLS, a method for variable selection in multiway data sets]{Sparse N-PLS, a method for variable selection in multiway data sets}



% ---------------------------------------------------------------------
% ---------------------------------------------------------------------
\section{Variable selection through L1 penalization}
As explained in \autoref{lasso}, L1-penalization forces the coefficients of a model to shrink to zero effectively performing variable selection. The original lasso for least squares is as follows:

\begin{equation}
2*x
\end{equation}

\section{Sparse N-PLS, integration of L1 penalization in the N-PLS algorithm}
\label{NPLSpenalization}
As commented above, we propose to introduce the L1-penalization in the N-PLS algorithm. To this aim, we used a similar approach to that of \cite{le2008sparse}. Briefly, to achieve sparse versions of $\textbf{\text{w}}^\text{J}$ and $\textbf{\text{w}}^\text{K}$ for each latent variable, we introduce the soft-thresholding penalty function $\beta_j^{lasso}=sgn(\beta_j^{LS})(|\beta_j^{LS}|-\lambda)^+$  defined in equation (6) in the N-PLS algorithm right after the SVD at the $\textbf{\text{w}}^\text{J}$ and $\textbf{\text{w}}^\text{K}$ determination. The complete algorithm is as follows:


\vspace{20pt}
Center X and Y, and unfold X (and Y when necessary) into a two-way matrix.

Let u be some column of Y, and set  f=1

\begin{enumerate}
    \item wT=uTX/uTu
    \item Build Z by refolding w according to the modes dimensions
    \item Determine wJ y wK by SVD
    \item L1-penalization inclusion
    \begin{enumerate}
        \item Apply soft-thresholding on wJ
        \item Apply soft-thresholding on wK
        \item Input the new w as kronecker(wK, wJ)
    \end{enumerate}
    \item t=Xw/wTw
    \item q=YTt/norm(YTt)
    \item u=Yq
    \item Check for convergence. If it is achieved, continue; otherwise, go to 1
    \item b = (TTT)-1TTu; where T=[t1 t2… tf]
    \item X = X-twT and Y = Y-tbqT
    \item f = f+1. Continue from step 1 until a good description of Y
\end{enumerate}
\vspace{20pt}

In this work, we performed both the standard regression (continuous response) and the discriminant version of the N-PLS model, i.e. N-PLS-DA. In the case of N-PLS-DA, Y is a y vector formed by ones and zeros, each of the two values related to one of the two classes to be segregated. 



\section{Parameters of the sparse N-PLS algorithm}
\subsection{Tuning of the parameters}